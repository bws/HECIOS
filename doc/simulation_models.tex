\documentclass[11pt]{article}
\usepackage{graphicx}
\usepackage{times}
\usepackage{pifont}
\usepackage{subfigure}
\usepackage{url}
\usepackage{listings}
\usepackage[T1]{fontenc}
\usepackage{ae,aecompl}

%
% GET THE MARGINS RIGHT, THE UGLY WAY (from PVFS)
%
\topmargin 0.2in
\textwidth 6.5in
\textheight 8.75in
\columnsep 0.25in
\oddsidemargin 0.0in
\evensidemargin 0.0in
\headsep 0.0in
\headheight 0.0in


%
% Enable figure subnumbering at the section level -- still doesn't work
% will figure this out when it becomes more important
%
\newcommand{\Section}[1]{\section{#1} \setcounter{figure}{1}}

%
% Actual document content begins here
%
\title{HECIOS: High End Computing I/O Simulator 
       (Nominations for a new name are welcome)}
\author{HECIOS Development Team}
\date{Fall 2006, Spring 2007}

\pagestyle{plain}
\begin{document}

\maketitle
\newpage

%\setcounter{tocdepth}{2}
%\tableofcontents
%\newpage

%
% These two give us paragraphs with space between
%
\setlength{\parindent}{0pt}
\setlength{\parskip}{11pt}

%
% Begin document body text
%
\section{Introduction to the Simulator}
HECIOS, the High End Computing I/O Simulator, is a trace driven simulator
designed to explore the I/O characteristics of very large cluster storage
systems.  The simulator leverages the OmNet++ simulation package to access an
event driven simulation package that provides existing well developed
networking components.

There are several things that we need to verify about the simulator and
OmNet++ before implementing the other portions of the simulator too
carefully.  Specifically, we need to make sure that the networking components
available within OmNet can be used by our simulation.  It would also be
beneficial to check on sticking the disk model stuff into our simulator.
Finally, we will need to make sure there is a working build system (Makefiles)
for integrating all of this code together in a working package.

\section{Simulator Overview}

\section{Simulator Layers}
HECIOS is architected using a layered approach.  The 4 layers are: the
configuration layer, the client storage layer, the server storage layer, and
the system layer.  This layering is similar to the high level layering used in
modern cluster file systems.
 
\begin{figure}[t]
  \begin{center}
    \includegraphics[scale=.6]{figs/model_layers}
  \end{center}
  \caption{Simulation Architectural Layers \label{fig:model-layers}}
\end{figure}

\subsection{Configuration Layer}
The configuration layer describes the experiment configuration.  Relevant
models within the configuration layer include the cluster type, the node
types, the number of cpus per node, and the number of I/O nodes in the system.
Additional parameters would include the amount of existing network, the
existing job mix, and the I/O application trace we are attempting to
benchmark.  All of these parameters need to be easily configurable, as these
parameters will generally be important in measuring a particular filesystem
configuration.

\subsubsection{ClusterConfig Model}
A model describing the number of compute nodes and I/O nodes of the cluster.
All cluster configuration models support 

\begin{itemize}
\item Set the number of compute nodes
\item Set the number of I/O nodes
\item Set the compute node network type (speed, topology, etc.)
\item Set the I/O node network type (speed, topology, etc.)
\item Set some type of utilization factor?? -- this should probably be set in
  the application, as my idea is that you specify real world existing clusters
  here, and then reuse them (Jazz, Adenine, LLNL BG/L, etc).  But setting it
  here maintains better encapsulation.
\end{itemize}

\subsubsection{ApplicationConfig Model}
A model describing the applications I/O trace and the number of dedicated
compute nodes.  This model will need to provide support for parsing multiple
trace formats, and possibly have a mechanism for allowing the generation of
random I/O trace information.  All Application Configurations support the
following operations:

\begin{itemize}
\item Set the number of job compute nodes
\item Set the number of job I/O nodes
\item Set the application/IO trace
\end{itemize}

\subsection{Client Storage Layer}
The following simulation models simulate the behaviour of the storage
interface available to applications running on a compute node.

\subsubsection{I/O Middleware}
Model to provide support for the various I/O primitives available to parallel
applications.  Should probably provide support for at least the following
serial file operations:  create, delete, open, stat, read, write.  Possibly 
sync.  Should also provide support for parallel I/O ops: open, read, write.
Maybe some stuff for working with data types/views.  Could be interesting.

\subsubsection{I/O Statistics}
A sensing layor for collecting statistics on client I/O operations.  I'm not
precisely sure where this will hook in, maybe it has hooks everywhere.

\subsubsection{DataDistribution}
The data distribution model describes how the client's logical data is mapped
to the storage servers.  Since we are currently only interested in simulating
file systems, this model is likely to have a bias for file systems; however, it
is not our intent to specify this model in a file system specific manner.

A data distribution fundamentally provides a list of I/O servers, and an
algorithm for mapping an application's logical data to a physical storage
location.  The data distribution provides the following model interfaces:

\begin{itemize}
\item Get the number of I/O servers
\item Get I/O server(s) for logical data location
\end{itemize}

In the file system specific case we expect the data locations to be logical 
file offsets.  Additionally, it may be important for a file distribution
model to handle multiple data locations simultaneously in order to perform
some data packing operations.

\subsubsection{ClientCache}
A model to provide caching semantics.  Almost definitely needs to be a read
through cache.  Write back versus write through should probably be
configurable.  Also needs to address the notions of caching things other than
data (metadata, handles, etc.)

\begin{itemize}
\item Read location
\item Read-through location
\item Write location
\item Invalidate range
\item Evict range
\end{itemize}

There are lots of questions about how to best organize this cache, since it
will probably be capable of storing variable length data.  A very non-cachey
idea.  Lots of space for someone to get really into it, and do a bunch of
background research and propose something I bet.

\subsubsection{NetworkTransport}
The storage layer network model.  This simply needs to provide a clean
interface for grabbing the detailed network stuff stored at the system layer.

\begin{itemize}
\item Send Request
\item Receive Response
\item Send Data
\item Recieve Data
\end{itemize}

\subsection{Server Storage Layer}
The following models simulate the server storage interface.

\subsubsection{Request Scheduler}
The queue for scheduling incoming operations.  This is a fairly critical
component as it pretty much defines the consistency mehanics for the file.

\begin{itemize}
\item Add Request
\end{itemize}

The only external operation I can think of is Add Request . . . nevertheless,
the request scheduler has alot of internal mechanics.  It needs to figure out
when queued requests can be performed.  Which is quite impacted by what
requests are currently being performed.  Some meta-data operations need to be
atomic (renames, e.g.), while I/O operations can hopefully be overlapped to a
large degree.  Additionally, there is need for fast algorithm for detecting
overlapping reads and writes.  Another valid research problem available for
poaching.

\subsection{System Layer}
The layer responsible for providing system level services: networking,
storage, etc.

\subsubsection{O/S (filesystem)}
I'm not too certain what needs to go here.  I suppose it is the equivalent of
trove in PVFS2.  So then something capable of simulating direct IO, AIO,
Berkely DB.  Etc.  Where should this be, should this be a slightly different
idea.  Walt will probably have strong opinions on this, I don't of yet.

\subsubsection{BlockCache}
A block device that acts as a cache.

\subsubsection{Disk}
A block device used to store data.

\subsubsection{Network}
A character device.  But probably higher level than that, most likely a packet
oriented device.  If it supported  DMA, it would be even better, as that is
the future according to many experts.

\end{document}

