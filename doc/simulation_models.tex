\documentclass[11pt]{article}
\usepackage[dvips]{graphicx}
\usepackage{times}

%
% GET THE MARGINS RIGHT, THE UGLY WAY
%
\topmargin 0.2in
\textwidth 6.5in
\textheight 8.75in
\columnsep 0.25in
\oddsidemargin 0.0in
\evensidemargin 0.0in
\headsep 0.0in
\headheight 0.0in


% \addtolength{\hoffset}{-2cm}
% \addtolength{\textwidth}{4cm}
%
% \addtolength{\voffset}{-1.5cm}
% \addtolength{\textheight}{3cm}
%

\title{HECIOS: High End Computing I/O Simulator (Nominations for names are welcome}
\author{HECIOS Development Team}
\date{Fall 2006, Spring 2007}

\pagestyle{plain}
\begin{document}

\maketitle

\newpage

\setcounter{tocdepth}{2}
\tableofcontents

%
% These two give us paragraphs with space between, which personally I
% think is the right way to have things.
%
\setlength{\parindent}{0pt}
\setlength{\parskip}{11pt}

%
% Begin document body text
%
\section{Introduction to the Simulator}
HECIOS, the High End Computing I/O Simulator, is a trace driven simulator
designed to explore the I/O characteristics of very large cluster storage
systems.  The simulator leverages the OmNet++ simulation package to access an
event driven simulation package that provides existing well developed
networking components.

\section{Simulator Overview}

\section{Simulator Models}
Note that the simulation models are simply the high level interfaces.  It
should be possible to create models within the named models below.

\subsection{System Models}

\subsubsection{ClusterConfig}
A model describing the number of compute nodes and I/O nodes of the cluster.

\subsubsection{ApplicationConfig}
A model describing the applications I/O trace and the number of dedicated
compute nodes.

\subsection{Client Filesystem Models}

\subsubsection{I/O Middleware}

\subsubsection{I/O Statistics}

\subsubsection{ClientCache}

\subsubsection{FileDistribution}

\subsubsection{NetworkTransport}
The network model.  And example of where a single high level interface may
stand as a proxy for alot of detail underneath.

\subsection{Server Filesystem Models}

\subsubsection{Request Scheduler}

\subsubsection{O/S (filesystem)}

\subsubsection{BlockCache}

\subsubsection{Disk}


\end{document}

