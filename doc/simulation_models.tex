\documentclass[11pt]{article}
\usepackage{graphicx}
\usepackage{times}
\usepackage{pifont}
\usepackage{subfigure}
\usepackage{url}
\usepackage{listings}
\usepackage[T1]{fontenc}
\usepackage{ae,aecompl}

%
% GET THE MARGINS RIGHT, THE UGLY WAY
%
\topmargin 0.2in
\textwidth 6.5in
\textheight 8.75in
\columnsep 0.25in
\oddsidemargin 0.0in
\evensidemargin 0.0in
\headsep 0.0in
\headheight 0.0in


% \addtolength{\hoffset}{-2cm}
% \addtolength{\textwidth}{4cm}
%
% \addtolength{\voffset}{-1.5cm}
% \addtolength{\textheight}{3cm}
%

\title{HECIOS: High End Computing I/O Simulator 
       (Nominations for a new name are welcome)}
\author{HECIOS Development Team}
\date{Fall 2006, Spring 2007}

\pagestyle{plain}
\begin{document}

\maketitle
\newpage

\setcounter{tocdepth}{2}
\tableofcontents

%
% These two give us paragraphs with space between, which personally I
% think is the right way to have things.
%
\setlength{\parindent}{0pt}
\setlength{\parskip}{11pt}

%
% Begin document body text
%
\section{Introduction to the Simulator}
HECIOS, the High End Computing I/O Simulator, is a trace driven simulator
designed to explore the I/O characteristics of very large cluster storage
systems.  The simulator leverages the OmNet++ simulation package to access an
event driven simulation package that provides existing well developed
networking components.

\section{Simulator Overview}

\section{Simulator Models}
Note that the simulation models are simply the high level interfaces.  It
should be possible to create models within the named models below.

HECIOS is architected using a layered approach.  The 3 layers are: the system
layer, the client storage layer, and the server storage layer.  This layering
is similar to the layering used in modern cluster file systems.
 
\begin{figure}[t]
  \begin{center}
    \includegraphics[scale=.6]{figs/model_layers}
  \end{center}
  \caption{Simulation Model Layers \label{fig:model-layers}}
\end{figure}

\subsection{System Layer}
The system layer describes the experiment configuration.  The cluster type.
The node types.  The cpus per node.  The number of I/O nodes in the system.
The amount of existing network traffic.  The network type.  The connection
topology.  All the physical stuff that impacts I/O performance.

\subsubsection{ClusterConfig}
A model describing the number of compute nodes and I/O nodes of the cluster.

\begin{itemize}
\item Set the number of compute nodes
\item Set the number of I/O nodes
\item Set the compute node network type (speed, topology, etc.)
\item Set the I/O node network type (speed, topology, etc.)
\item Set some type of utilization factor??
\end{itemize}

\subsubsection{ApplicationConfig}
A model describing the applications I/O trace and the number of dedicated
compute nodes.  This model will need to provide support for parsing multiple
trace formats, and possibly have a mechanism for allowing the generation of
random I/O trace information.

\begin{itemize}
\item Set the number of job compute nodes
\item Set the number of job I/O nodes
\item Set the application/IO trace
\end{itemize}

\subsection{Client Storage Layer}
The following simulation models simulate the behaviour of the storage
interface available to applications running on a compute node.

\subsubsection{I/O Middleware}
Model to provide support for the various I/O primitives available to parallel
applications.  Should probably provide support for at least the following
serial file operations:  create, delete, open, stat, read, write.  Possibly 
sync.  Should also provide support for parallel I/O ops: open, read, write.
Maybe some stuff for working with data types/views.  Could be interesting.

\subsubsection{I/O Statistics}
A sensing layor for collecting statistics on client I/O operations.  I'm not
precisely sure where this will hook in, maybe it has hooks everywhere.

\subsubsection{DataDistribution}
The data distribution model describes how the client's logical data is mapped
to the storage servers.  Since we are currently only interested in simulating
file systems, this model is likely to have a bias for file systems; however, it
is not our intent to specify this model in a file system specific manner.

A data distribution fundamentally provides a list of I/O servers, and an
algorithm for mapping an application's logical data to a physical storage
location.  The data distribution provides the following model interfaces:

\begin{itemize}
\item Get the number of I/O servers
\item Get I/O server(s) for logical data location
\end{itemize}

In the file system specific case we expect the data locations to be logical 
file offsets.  Additionally, it may be important for a file distribution
model to handle multiple data locations simultaneously in order to perform
some data packing operations.

\subsubsection{ClientCache}
A model to provide caching semantics.  Almost definitely needs to be a read
through cache.  Write back versus write through should probably be
configurable.  Also needs to address the notions of caching things other than
data (metadata, handles, etc.)

Possible operations:  Read, Read-Through, Write, Invalidate, Evict.

\subsubsection{NetworkTransport}
The network model.  And example of where a single high level interface may
stand as a proxy for alot of detail underneath.

\subsection{Server Storage Layer}
The following models simulate the server storage interface.

\subsubsection{Request Scheduler}
The queue for scheduling incoming operations.  This is a fairly critical
component as it pretty much defines the consistency mehanics for the file.

\subsubsection{O/S (filesystem)}
I'm not too certain what needs to go here.  I suppose it is the equivalent of
trove in PVFS2.  So then something capable of simulating direct IO, AIO,
Berkely DB.  Etc.

\subsubsection{BlockCache}
A block device that acts as a cache.

\subsubsection{Disk}
A block device.

\end{document}

