\documentclass[11pt]{article}
\usepackage[dvips]{graphicx}
\usepackage{times}

%
% GET THE MARGINS RIGHT, THE UGLY WAY
%
\topmargin 0.2in
\textwidth 6.5in
\textheight 8.75in
\columnsep 0.25in
\oddsidemargin 0.0in
\evensidemargin 0.0in
\headsep 0.0in
\headheight 0.0in


% \addtolength{\hoffset}{-2cm}
% \addtolength{\textwidth}{4cm}
%
% \addtolength{\voffset}{-1.5cm}
% \addtolength{\textheight}{3cm}
%

\title{HECIOS: High End Computing I/O Simulator (Nominations for names are welcome}
\author{HECIOS Development Team}
\date{Fall 2006, Spring 2007}

\pagestyle{plain}
\begin{document}

\maketitle
\newpage

\setcounter{tocdepth}{2}
\tableofcontents

%
% These two give us paragraphs with space between, which personally I
% think is the right way to have things.
%
\setlength{\parindent}{0pt}
\setlength{\parskip}{11pt}

%
% Begin document body text
%
\section{Introduction to the Simulator}
HECIOS, the High End Computing I/O Simulator, is a trace driven simulator
designed to explore the I/O characteristics of very large cluster storage
systems.  The simulator leverages the OmNet++ simulation package to access an
event driven simulation package that provides existing well developed
networking components.

\section{Simulator Overview}

\section{Simulator Models}
Note that the simulation models are simply the high level interfaces.  It
should be possible to create models within the named models below.

\begin{figure}[t]
  \begin{center}
    \includegraphics[scale=.8]{figs/model_layers}
  \end{center}
  \caption{Simulation Model Layers \label{fig:model-layers}}
\end{figure}

\subsection{System Models}
System models control the experiment configuration.  The cluster type.  The
node types.  The cpus per node.  The number of I/O nodes in the system.  The amount
of existing network traffic.  The network type.  The connection topology.  All
the physical stuff that impacts I/O performance.

\subsubsection{ClusterConfig}
A model describing the number of compute nodes and I/O nodes of the cluster.

\subsubsection{ApplicationConfig}
A model describing the applications I/O trace and the number of dedicated
compute nodes.

\subsection{Client Filesystem Models}
The following simulation models simulate the behaviour of the client's file
system interface.

\subsubsection{I/O Middleware}
Model to provide support for the various I/O primitives available to parallel
applications.  Should probably provide support for at least the following
serial file operations:  create, delete, open, stat, read, write.  Possibly 
sync.  Should also provide support for parallel I/O ops: open, read, write.
Maybe some stuff for working with data types/views.  Could be interesting.

\subsubsection{I/O Statistics}
A sensing layor for collecting statistics on client I/O operations.  I'm not
precisely sure where this will hook in, maybe it has hooks everywhere.

\subsubsection{ClientCache}
A model to provide caching semantics.  Almost definitely needs to be a read
through cache.  Write back versus write through should probably be
configurable.  Also needs to address the notions of caching things other than
data (metadata, handles, etc.)

\subsubsection{FileDistribution}
Describes how the file is distributed among the I/O servers.

\subsubsection{NetworkTransport}
The network model.  And example of where a single high level interface may
stand as a proxy for alot of detail underneath.

\subsection{Server Filesystem Models}
The following models simulate the filesystem's server interface.

\subsubsection{Request Scheduler}
The queue for scheduling incoming operations.  This is a fairly critical
component as it pretty much defines the consistency mehanics for the file.

\subsubsection{O/S (filesystem)}
I'm not too certain what needs to go here.  I suppose it is the equivalent of
trove in PVFS2.  So then something capable of simulating direct IO, AIO,
Berkely DB.  Etc.

\subsubsection{BlockCache}
A block device that acts as a cache.

\subsubsection{Disk}
A block device.

\end{document}

