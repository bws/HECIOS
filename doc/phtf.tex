\documentclass[9pt]{article}
\title{The Definition of PHTF (Parallel Hecios Trace Format)}
\author{Yang Wu}

\begin{document}
\maketitle

\section{Instrduction}
\label{sec:Instrduction}

HTF consists of three parts:
\begin{enumerate}
	\item Machine Archetecture (Meta)
	\item File System State (Meta)
	\item Record of Events (Data)
\end{enumerate}

All files from one trace reside in one folder, should look like:
\begin{verbatim}
[.] [..] arch.ini fs.ini event.0 ... event.n
\end{verbatim}

\begin{description}
	\item[arch.ini - the machine archtecture description file] 
	\item[fs.ini - the file system state description file]
	\item[event.k - the events record file, k denotes the id of process.]
\end{description}

\section{The arch.ini File}
\label{sec:TheArchIniFile}

The arch file contains a description of the architecture information. This file adopts the '.ini' file syntax, which consists several sections and many fields in each section.

\subsection{Sample}
\label{sec:SampleArch}
\begin{verbatim}
[Cluster]
Comp Num=64
I/O Num=128
Network Arch=flat

[Computer Node]
Host OS=linux
CPU=intel piii
RAM=1024
Network Arch=...
...
\end{verbatim}

\subsection{Section List}
\label{sec:SectionList}

\subsubsection{Cluster}
\label{sec:Cluster}
\begin{center}
	\begin{tabular}{l | c | l}
		Field & Data Type & Description\\\hline
		Comp Num & long & Number of compute nodes\\
		I/O Num & long & Number of I/O nodes\\
		Network Arch & string & Interconnection network architecture (flat, tree, etc)\\
	\end{tabular}
\end{center}

\subsubsection{Compute Node}
\label{sec:ComputeNode}
\begin{center}
	\begin{tabular}{l | c | l}
		Field & Data Type & Description\\\hline
		Host OS & string & Operating system\\
		CPU & string & \\
		RAM & long & RAM (MB)\\
		Network Arch & string & ...\\
		Msg Cap & long & Simultaneous messaging capability\\		
	\end{tabular}
\end{center}

\subsubsection{I/O Node}
\label{sec:IONode}
\begin{center}
	\begin{tabular}{l | c | l}
		Field & Data Type & Description\\\hline
		Host OS & string & Operating system\\
		CPU & string & \\
		RAM & long & RAM (MB) \\
		Network Arch & string & ...\\
		Msg Cap & long & Simultaneous messaging capability\\
	\end{tabular}
\end{center}

\subsubsection{Storage}
\label{sec:Storage}
\begin{center}
	\begin{tabular}{l | c | l}
		Field & Data Type & Description\\\hline
		Throuput & long & ... \\
		Latency & long & ... \\
		Interface & string & (scsi, fibre, sata, etc.)\\
		Reducdancy & long & ...\\
		Paral FS & string & Resident parallel file system in use\\
		Native FS & string & Resident native file system in use \\
		Cache & string & disk, os \\
	\end{tabular}
\end{center}

\subsubsection{Disk}
\label{sec:Disk}
\begin{center}
	\begin{tabular}{l | c | l}
		Field & Data Type & Description\\\hline
		RPM & long & ...\\
		Cache & long & (MB)\\
		Settle Time & long & ...\\
		Accel & long & Acceleration in sectors\\
		Time to Max & long & Time to max rpm \\
		Cylinders & long & ...\\
		Tracks & long & ...\\
		Sectors & long & ...\\
	\end{tabular}
\end{center}

\section{The fs.ini File}
\label{sec:TheFsIniFile}
TODO: Add content

\section{The event.k File}
\label{sec:TheEventKFile}
Generally speaking, the event file is a set of plain text files, each denotes a process and contains several records of events.
\subsection{The Syntax}
\label{sec:TheSyntax}
\begin{verbatim}
<event file> ::= <record> | <record><record>
<record> ::= <id> <op> <start time> <duration> <return value> <parameters> <eol>
<id> ::= <number>
<op> ::= <mpi calls> | <special calls>
<start time> ::= <number>
<duration> ::= <number>
<return value> ::= NULL | <string> <return value> | <number> <return value>
<parameters> ::= NULL | <string> <parameters> | <number> <parameters>
<eol> ::= "\r\n"
<mpi calls> ::= {MPI-I/Os}
<special calls> ::= PAUSE call
<number> ::= <digit><number>
<string> ::= <character><string> | <digit><string>
<digit> ::= {0123456789}
<character> ::= {a-zA-Z_/.}
\end{verbatim}

\subsection{Description}
\label{sec:Description}
\begin{description}
	\item[record - one event that happened on current process]\footnote{The process that this file denotes}
	\item[id - denotes the sequence of execution]
	\item[op - type of operation(MPI-I/O calls or PAUSE call)]\footnote{A PAUSE call ask the process to pause for a certain time}
	\item[start time - the start time of the original operation for reference]
	\item[duration - the duration of the original operation for reference]
        \item[return value - the return value of the original operation]
	\item[parameters - the parameters of the original operation]\footnote{Certain data type as point to string would be translated into data it points to}
\end{description}

\subsection{Sample}
\label{sec:SampleMPI}
\begin{verbatim}
1 MPI_FILE_READ 12341 25 6 0x12dfe123 500 0 0x12dff1229
2 PAUSE 12366 30
3 MPI_FILE_CLOSE 12396 5 6
4 ...
...
\end{verbatim}

\subsection{PAUSE Call}\footnote{The parameters in event description follow a 'DATATYPE\_DESCRIPTION' syntax. For example, 'l\_start\_time' represents a long integer that holds the start time of an event.}
\label{sec:PAUSECall}
\begin{verbatim}
Event:
 PAUSE <l_start_time> <l_duration>
\end{verbatim}

\subsection{MPI-I/O List}
\label{sec:MPIIOList}

\subsubsection{MPI\_FILE\_OPEN}
\label{sec:MPIFILEOPEN}
\begin{verbatim}
Prototype: 
 int MPI_FILE_OPEN(IN comm, IN filename, IN amode, IN info, OUT fh)
Event: 
 MPI_FILE_READ <l_start_time> <l_duration> <l_return> 
               <l_comm_id> <str_fn> <l_amode> 
               <ptr_info> <ptr_fh>
\end{verbatim}

\subsubsection{MPI\_FILE\_CLOSE}
\label{sec:MPIFILECLOSE}
\begin{verbatim}
Prototype: 
 int MPI_FILE_CLOSE(INOUT fh)
Event: 
 MPI_FILE_CLOSE <l_start_time> <l_duration> <l_return> 
                <ptr_fh>
\end{verbatim}

\subsubsection{MPI\_FILE\_DELETE}
\label{sec:MPIFILEDELETE}
\begin{verbatim}
Prototype: 
 int MPI_FILE_DELETE(IN filename)
Event: 
 MPI_FILE_DELETE <l_start_time> <l_duration> <l_return>
                 <str_fn>
\end{verbatim}

\subsubsection{MPI\_FILE\_SET\_SIZE}
\label{sec:MPIFILESETSIZE}
\begin{verbatim}
Prototype: 
 int MPI_FILE_SET_SIZE(INOUT fh, IN size)
Event: 
 MPI_FILE_SET_SIZE <l_start_time> <l_duration> <l_return>
                   <ptr_fh> <l_size>
\end{verbatim}

\subsubsection{MPI\_FILE\_PREALLOCATE}
\label{sec:MPIFILEPREALLOCATE}
\begin{verbatim}
Prototype: 
 int MPI_FILE_PREALLOCATE(INOUT fh, IN size)
Event: 
 MPI_FILE_PREALLOCATE <l_start_time> <l_duration> <l_return>
                      <ptr_fh> <l_size>
\end{verbatim}

\subsubsection{MPI\_FILE\_GET\_SIZE}
\label{sec:MPIFILEGETSIZE}
\begin{verbatim}
Prototype:
 int MPI_FILE_GET_SIZE(IN fh, OUT size)
Event:
 MPI_FILE_GET_SIZE <l_start_time> <l_duration> <l_return>
                   <ptr_fh> <l_size>
\end{verbatim}

\subsubsection{MPI\_FILE\_GET\_GROUP}
\label{sec:MPIFILEGETGROUP}
\begin{verbatim}
Prototype:
 int MPI_FILE_GET_GROUP(IN fh, OUT group)
Event:
 MPI_FILE_GET_GROUP <l_start_time> <l_duration> <l_return>
                    <ptr_fh> <ptr_grp>
\end{verbatim}

\subsubsection{MPI\_FILE\_GET\_AMODE}
\label{sec:MPIFILEGETAMODE}
\begin{verbatim}
Prototype:
 int MPI_FILE_GET_AMODE(IN fh, OUT amode)
Event:
 MPI_FILE_GET_AMODE <l_start_time> <l_duration> <l_return>
                    <ptr_fh> <l_amode>
\end{verbatim}

\subsubsection{MPI\_FILE\_SET\_INFO}
\label{sec:MPIFILESETINFO}
\begin{verbatim}
Prototype:
 int MPI_FILE_SET_INFO(IN fh, IN info)
Event:
 MPI_FILE_SET_INFO <l_start_time> <l_duration> <l_return>
                   <ptr_fh> <ptr_info>
\end{verbatim}

\subsubsection{MPI\_FILE\_GET\_INFO}
\label{sec:MPIFILEGETINFO}
\begin{verbatim}
Prototype:
 int MPI_FILE_GET_INFO(IN fh, OUT info)
Event:
 MPI_FILE_GET_INFO <l_start_time> <l_duration> <l_return>
                   <ptr_fh> <ptr_info>
\end{verbatim}

\subsubsection{MPI\_FILE\_SET\_VIEW}
\label{sec:MPIFILESETVIEW}
\begin{verbatim}
Prototype:
 int MPI_FILE_SET_VIEW(INOUT fh, IN disp, IN etype, IN filetype, 
                   IN datarep, IN info)
Event:
 MPI_FILE_SET_VIEW <l_start_time> <l_duration> <l_return>
                   <ptr_fh> <l_disp> <ptr_et> 
                   <ptr_ft> <str_dr> <ptr_info>
\end{verbatim}

\subsubsection{MPI\_FILE\_READ\_AT}
\label{sec:MPIFILEREADAT}
\begin{verbatim}
Prototype:
 int MPI_FILE_READ_AT(IN fh, IN offset, OUT buf, IN count, 
                  IN datatype, OUT status)
Event:
 MPI_FILE_READ_AT <l_start_time> <l_duration> <l_return>
                  <ptr_fh> <l_offset> <ptr_buf>
                  <l_count> <ptr_dt> <ptr_st>
\end{verbatim}

\subsubsection{MPI\_FILE\_READ\_AT\_ALL}
\label{sec:MPIFILEREADATALL}
\begin{verbatim}
Prototype:
 int MPI_FILE_READ_AT_ALL(IN fh, IN offset, OUT buf, IN count, 
                      IN datatype, OUT status)
Event:
 MPI_FILE_READ_AT_ALL <l_start_time> <l_duration> <l_return>
                      <ptr_fh> <l_offset> <ptr_buf>
                      <l_count> <ptr_dt> <ptr_st>
\end{verbatim}

\subsubsection{MPI\_FILE\_WRITE\_AT}
\label{sec:MPIFILEWRITEAT}
\begin{verbatim}
Prototype:
 int MPI_FILE_WRITE_AT(IN fh, IN offset, IN buf, IN count, 
                   IN datatype, OUT status)
Event:
 MPI_FILE_WRITE_AT <l_start_time> <l_duration> <l_return>
                   <ptr_fh> <l_offset> <ptr_buf>
                   <l_count> <ptr_dt> <ptr_st>
\end{verbatim}

\subsubsection{MPI\_FILE\_WRITE\_AT\_ALL}
\label{sec:MPIFILEWRITEATALL}
\begin{verbatim}
Prototype:
 int MPI_FILE_WRITE_AT_ALL(IN fh, IN offset, IN buf, IN count, 
                       IN datatype, OUT status)
Event:
 MPI_FILE_WRITE_AT_ALL <l_start_time> <l_duration> <l_return>
                       <ptr_fh> <l_offset> <ptr_buf>
                       <l_count> <ptr_dt> <ptr_st>
\end{verbatim}

\subsubsection{MPI\_FILE\_IREAD\_AT}
\label{sec:MPIFILEIREADAT}
\begin{verbatim}
Prototype:
 int MPI_FILE_IREAD_AT(IN fh, IN offset, OUT buf, IN count, 
                   IN datatype, OUT status)
Event:
 MPI_FILE_IREAD_AT <l_start_time> <l_duration> <l_return>
                   <ptr_fh> <l_offset> <ptr_buf>
                   <l_count> <ptr_dt> <ptr_st>
\end{verbatim}

\subsubsection{MPI\_FILE\_IWRITE\_AT}
\label{sec:MPIFILEIWRITEAT}
\begin{verbatim}
Prototype:
 int MPI_FILE_IWRITE_AT(IN fh, IN offset, IN buf, IN count, 
                    IN datatype, OUT status)
Event:
 MPI_FILE_IWRITE_AT <l_start_time> <l_duration> <l_return>
                    <ptr_fh> <l_offset> <ptr_buf>
                    <l_count> <ptr_dt> <ptr_st>
\end{verbatim}

\subsubsection{MPI\_FILE\_READ}
\label{sec:MPIFILEREAD}
\begin{verbatim}
Prototype:
 int MPI_FILE_READ(INOUT fh, OUT buf, IN count, IN datatype, 
               OUT status)
Event:
 MPI_FILE_READ <l_start_time> <l_duration> <l_return>
               <ptr_fh> <ptr_buf> <l_count>
               <ptr_dt> <ptr_st>
\end{verbatim}

\subsubsection{MPI\_FILE\_READ\_ALL}
\label{sec:MPIFILEREADALL}
\begin{verbatim}
Prototype:
 int MPI_FILE_READ_ALL(INOUT fh, OUT buf, IN count, IN datatype, 
                   OUT status)
Event:
 MPI_FILE_READ_ALL <l_start_time> <l_duration> <l_return>
                   <ptr_fh> <ptr_buf> <l_count>
                   <ptr_dt> <ptr_st>
\end{verbatim}

\subsubsection{MPI\_FILE\_WRITE}
\label{sec:MPIFILEWRITE}
\begin{verbatim}
Prototype:
 int MPI_FILE_WRITE(INOUT fh, IN buf, IN count, IN datatype, 
                OUT status)
Event:
 MPI_FILE_WRITE <l_start_time> <l_duration> <l_return>
                <ptr_fh> <ptr_buf> <l_count>
                <ptr_dt> <ptr_st>
\end{verbatim}

\subsubsection{MPI\_FILE\_WRITE\_ALL}
\label{sec:MPIFILEWRITEALL}
\begin{verbatim}
Prototype:
 int MPI_FILE_WRITE_ALL(INOUT fh, IN buf, IN count, IN datatype, 
                    OUT status)
Event:
 MPI_FILE_WRITE_ALL <l_start_time> <l_duration> <l_return>
                    <ptr_fh> <ptr_buf> <l_count>
                    <ptr_dt> <ptr_st>
\end{verbatim}

\subsubsection{MPI\_FILE\_IREAD}
\label{sec:MPIFILEIREAD}
\begin{verbatim}
Prototype:
 int MPI_FILE_IREAD(INOUT fh, OUT buf, IN count, IN datatype, 
                OUT status)
Event:
 MPI_FILE_IREAD <l_start_time> <l_duration> <l_return>
                <ptr_fh> <ptr_buf> <l_count>
                <ptr_dt> <ptr_st>
\end{verbatim}

\subsubsection{MPI\_FILE\_IWRITE}
\label{sec:MPIFILEIWRITE}
\begin{verbatim}
Prototype:
 int MPI_FILE_IWRITE(INOUT fh, IN buf, IN count, IN datatype, 
                 OUT status)
Event:
 MPI_FILE_IWRITE <l_start_time> <l_duration> <l_return>
                 <ptr_fh> <ptr_buf> <l_count>
                 <ptr_dt> <ptr_st>
\end{verbatim}

\subsubsection{MPI\_FILE\_SEEK}
\label{sec:MPIFILESEEK}
\begin{verbatim}
Prototype:
 int MPI_FILE_SEEK(INOUT fh, IN offset, IN whence)
Event:
 MPI_FILE_SEEK <l_start_time> <l_duration> <l_return>
               <ptr_fh> <l_offset> <l_whence>
\end{verbatim}

\subsubsection{MPI\_FILE\_GET\_POSITION}
\label{sec:MPIFILEGETPOSITION}
\begin{verbatim}
Prototype:
 int MPI_FILE_GET_POSITION(IN fh, OUT offset)
Event:
 MPI_FILE_GET_POSITION <l_start_time> <l_duration> <l_return>
                       <ptr_fh> <l_offset>
\end{verbatim}

\subsubsection{MPI\_FILE\_GET\_BYTE\_OFFSET}
\label{sec:MPIFILEGETBYTEOFFSET}
\begin{verbatim}
Prototype:
 int MPI_FILE_GET_BYTE_OFFSET(IN fh, IN offset, OUT disp)
Event:
 MPI_FILE_GET_BYTE_OFFSET <l_start_time> <l_duration> <l_return>
                          <ptr_fh> <l_offset>
\end{verbatim}

\subsubsection{MPI\_FILE\_READ\_SHARED}
\label{sec:MPIFILEREADSHARED}
\begin{verbatim}
Prototype:
 int MPI_FILE_READ_SHARED(INOUT fh, OUT buf, IN count, 
                      IN datatype, OUT status)
Event:
 MPI_FILE_READ_SHARED <l_start_time> <l_duration> <l_return>
                      <ptr_fh> <ptr_buf> <l_count>
                      <ptr_dt> <ptr_st>
\end{verbatim}

\subsubsection{MPI\_FILE\_WRITE\_SHARED}
\label{sec:MPIFILEWRITESHARED}
\begin{verbatim}
Prototype:
 int MPI_FILE_WRITE_SHARED(INOUT fh, IN buf, IN count, 
                       IN datatype, OUT status)
Event:
 MPI_FILE_WRITE_SHARED <l_start_time> <l_duration> <l_return>
                       <ptr_fh> <ptr_buf> <l_count>
                       <ptr_dt> <ptr_st>
\end{verbatim}

\subsubsection{MPI\_FILE\_IREAD\_SHARED}
\label{sec:MPIFILEIREADSHARED}
\begin{verbatim}
Prototype:
 int MPI_FILE_IREAD_SHARED(INOUT fh, OUT buf, IN count, 
                       IN datatype, OUT status)
Event:
 MPI_FILE_IREAD_SHARED <l_start_time> <l_duration> <l_return>
                       <ptr_fh> <ptr_buf> <l_count>
                       <ptr_dt> <ptr_st>
\end{verbatim}

\subsubsection{MPI\_FILE\_IWRITE\_SHARED}
\label{sec:MPIFILEIWRITESHARED}
\begin{verbatim}
Prototype:
 int MPI_FILE_IWRITE_SHARED(INOUT fh, IN buf, IN count, 
                        IN datatype, OUT status)
Event:
 MPI_FILE_IWRITE_SHARED <l_start_time> <l_duration> <l_return>
                        <ptr_fh> <ptr_buf> <l_count>
                        <ptr_dt> <ptr_st>
\end{verbatim}

\subsubsection{MPI\_FILE\_READ\_ORDERED}
\label{sec:MPIFILEREADORDERED}
\begin{verbatim}
Prototype:
 int MPI_FILE_READ_ORDERED(INOUT fh, OUT buf, IN count, 
                       IN datatype, OUT status)
Event:
 MPI_FILE_READ_ORDERED <l_start_time> <l_duration> <l_return>
                       <ptr_fh> <ptr_buf> <l_count>
                       <ptr_dt> <ptr_st>
\end{verbatim}

\subsubsection{MPI\_FILE\_WRITE\_ORDERED}
\label{sec:MPIFILEWRITEORDERED}
\begin{verbatim}
Prototype:
 int MPI_FILE_WRITE_ORDERED(INOUT fh, IN buf, IN count, 
                        IN datatype, OUT status)
Event:
 MPI_FILE_WRITE_ORDERED <l_start_time> <l_duration> <l_return>
                        <ptr_fh> <ptr_buf> <l_count>
                        <ptr_dt> <ptr_st>
\end{verbatim}

\subsubsection{MPI\_FILE\_SEEK\_SHARED}
\label{sec:MPIFILESEEKSHARED}
\begin{verbatim}
Prototype:
 int MPI_FILE_SEEK_SHARED(INOUT fh, IN offset, IN whence)
Event:
 MPI_FILE_SEEK_SHARED <l_start_time> <l_duration> <l_return>
                      <ptr_fh> <l_offset> <l_whence>
\end{verbatim}

\subsubsection{MPI\_FILE\_GET\_POSITION\_SHARED}
\label{sec:MPIFILEGETPOSITIONSHARED}
\begin{verbatim}
Prototype:
 int MPI_FILE_GET_POSITION_SHARED(IN fh, OUT offset)
Event:
 MPI_FILE_GET_POSITION_SHARED <l_start_time> <l_duration> <l_return>
                              <ptr_fh> <l_offset>
\end{verbatim}

\subsubsection{MPI\_FILE\_READ\_AT\_ALL\_BEGIN}
\label{sec:MPIFILEREADATALLBEGIN}
\begin{verbatim}
Prototype:
 int MPI_FILE_READ_AT_ALL_BEGIN(IN fh, IN offset, OUT buf, IN count, IN datatype)
Event:
 MPI_FILE_READ_AT_ALL_BEGIN <l_start_time> <l_duration> <l_return>
                            <ptr_fh> <l_offset> <l_count>
                            <ptr_dt>
\end{verbatim}

\subsubsection{MPI\_FILE\_READ\_AT\_ALL\_END}
\label{sec:MPIFILEREADATALLEND}
\begin{verbatim}
Prototype:
 int MPI_FILE_READ_AT_ALL_END(IN fh, OUT buf, OUT status)
Event:
 MPI_FILE_READ_AT_ALL_END <l_start_time> <l_duration> <l_return>
                          <ptr_fh> <ptr_buf> <ptr_st>
\end{verbatim}

\subsubsection{MPI\_FILE\_WRITE\_AT\_ALL\_BEGIN}
\label{sec:MPIFILEWRITEATALLBEGIN}
\begin{verbatim}
Prototype:
 int MPI_FILE_WRITE_AT_ALL_BEGIN(INOUT fh, IN offset, IN buf, IN count, IN datatype)
Event:
 MPI_FILE_WRITE_AT_ALL_BEGIN <l_start_time> <l_duration> <l_return>
                             <ptr_fh> <l_offset> <l_count>
                             <ptr_dt>
\end{verbatim}

\subsubsection{MPI\_FILE\_WRITE\_AT\_ALL\_END}
\label{sec:MPIFILEWRITEATALLEND}
\begin{verbatim}
Prototype:
 int MPI_FILE_WRITE_AT_ALL_END(INOUT fh, IN buf, OUT status)
Event:
 MPI_FILE_WRITE_AT_ALL_END <l_start_time> <l_duration> <l_return>
                           <ptr_fh> <ptr_buf> <ptr_st>
\end{verbatim}

\subsubsection{MPI\_FILE\_READ\_ALL\_BEGIN}
\label{sec:MPIFILEREADALLBEGIN}
\begin{verbatim}
Prototype:
 int MPI_FILE_READ_ALL_BEGIN(INOUT fh, OUT buf, IN count, IN datatype)
Event:
 MPI_FILE_READ_ALL_BEGIN <l_start_time> <l_duration> <l_return>
                         <ptr_fh> <ptr_buf> <l_count>
                         <ptr_dt>
\end{verbatim}

\subsubsection{MPI\_FILE\_READ\_ALL\_END}
\label{sec:MPIFILEREADALLEND}
\begin{verbatim}
Prototype:
 int MPI_FILE_READ_ALL_END(INOUT fh, OUT buf, OUT status)
Event:
 MPI_FILE_READ_ALL_END <l_start_time> <l_duration> <l_return>
                       <ptf_fh> <ptr_buf> <ptr_st>
\end{verbatim}

\subsubsection{MPI\_FILE\_WRITE\_ALL\_BEGIN}
\label{sec:MPIFILEWRITEALLBEGIN}
\begin{verbatim}
Prototype:
 int MPI_FILE_WRITE_ALL_BEGIN(INOUT fh, IN buf, IN count, IN datatype)
Event:
 MPI_FILE_WRITE_ALL_BEGIN <l_start_time> <l_duration> <l_return>
                          <ptr_fh> <ptr_buf> <l_count>
                          <ptr_dt>
\end{verbatim}

\subsubsection{MPI\_FILE\_WRITE\_ALL\_END}
\label{sec:MPIFILEWRITEALLEND}
\begin{verbatim}
Prototype:
 int MPI_FILE_WRITE_ALL_END(INOUT fh, IN buf, OUT status)
Event:
 MPI_FILE_WRITE_ALL_END <l_start_time> <l_duration> <l_return>
                        <ptf_fh> <ptr_buf> <ptr_st>
\end{verbatim}

\subsubsection{MPI\_FILE\_READ\_ORDERED\_BEGIN}
\label{sec:MPIFILEREADORDEREDBEGIN}
\begin{verbatim}
Prototype:
 int MPI_FILE_READ_ORDERED_BEGIN(INOUT fh, OUT buf, IN count, IN datatype)
Event:
 MPI_FILE_READ_ORDERED_BEGIN <l_start_time> <l_duration> <l_return>
                             <ptr_fh> <ptr_buf> <l_count>
                             <ptr_dt>
\end{verbatim}

\subsubsection{MPI\_FILE\_READ\_ORDERED\_END}
\label{sec:MPIFILEREADORDEREDEND}
\begin{verbatim}
Prototype:
 int MPI_FILE_READ_ORDERED_END(INOUT fh, OUT buf, OUT status)
Event:
 MPI_FILE_READ_ORDERED_END <l_start_time> <l_duration> <l_return>
                           <ptf_fh> <ptr_buf> <ptr_st>
\end{verbatim}

\subsubsection{MPI\_FILE\_WRITE\_ORDERED\_BEGIN}
\label{sec:MPIFILEWRITEORDEREDBEGIN}
\begin{verbatim}
Prototype:
 int MPI_FILE_WRITE_ORDERED_BEGIN(INOUT fh, IN buf, IN count, IN datatype)
Event:
 MPI_FILE_WRITE_ORDERED_BEGIN <l_start_time> <l_duration> <l_return>
                              <ptr_fh> <ptr_buf> <l_count>
                              <ptr_dt>
\end{verbatim}

\subsubsection{MPI\_FILE\_WRITE\_ORDERED\_END}
\label{sec:MPIFILEWRITEORDEREDEND}
\begin{verbatim}
Prototype:
 int MPI_FILE_WRITE_ORDERED_END(INOUT fh, IN buf, OUT status)
Event:
 MPI_FILE_WRITE_ORDERED_END <l_start_time> <l_duration> <l_return>
                            <ptf_fh> <ptr_buf> <ptr_st>
\end{verbatim}

\subsubsection{MPI\_FILE\_GET\_TYPE\_EXTENT}
\label{sec:MPIFILEGETTYPEEXTENT}
\begin{verbatim}
Prototype:
 int MPI_FILE_GET_TYPE_EXTENT(IN fh, IN datatype, OUT extent)
Event:
 MPI_FILE_GET_TYPE_EXTENT <l_start_time> <l_duration> <l_return>
                          <ptr_fh> <ptr_dt> <l_ext>
\end{verbatim}

\subsubsection{MPI\_FILE\_SET\_ATOMICITY}
\label{sec:MPIFILESETATOMICITY}
\begin{verbatim}
Prototype:
 int MPI_FILE_SET_ATOMICITY(IN fh, IN flag)
Event:
 MPI_FILE_SET_ATOMICITY <l_start_time> <l_duration> <l_return>
                        <ptr_fh> <b_flg>
\end{verbatim}

\subsubsection{MPI\_FILE\_GET\_ATOMICITY}
\label{sec:MPIFILEGETATOMICITY}
\begin{verbatim}
Prototype:
 int MPI_FILE_GET_ATOMICITY(IN fh, OUT flag)
Event:
 MPI_FILE_GET_ATOMICITY <l_start_time> <l_duration> <l_return>
                        <ptr_fh> <b_flg>
\end{verbatim}

\subsubsection{MPI\_FILE\_SYNC}
\label{sec:MPIFILESYNC}
\begin{verbatim}
Prototype:
 int MPI_FILE_SYNC(INOUT fh)
Event:
 MPI_FILE_SYNC <l_start_time> <l_duration> <l_return>
               <ptr_fh>
\end{verbatim}

\end{document}
